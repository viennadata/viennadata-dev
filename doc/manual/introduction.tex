
\chapter*{Introduction}   \addcontentsline{toc}{chapter}{Introduction} \label{intro}

\NOTE{This documentation hasn't been updated to the interface of the upcoming ViennaData 1.1.0 release yet}


When designing classes in object-oriented programming languages,
one has to distinguish between intrinsic properties which are inherent to the modelled entity,
and data which is associated with the entity. For example, a triangle is intrinsically defined
by three disjoint points, while associated data can be color or texture information in graphical applications, or material properties for two-dimensional scientific simulations.

While associated data depends on the respective application, intrinsic properties do not.
Therefore, when designing reusable classes, associated data must not be part of the class itself.
Reconsidering the example of a triangle, a class member for color information would immediately result in
a potentially significant overhead for scientific simulations and thus reduce the reuseability of the class.
However, the color information for the triangle still has to be accessible for graphical applications. This is where
{\ViennaData} comes into play, because it provides a convenient means to store and access associated data
for small, reusable classes with intrinsic data members only.

As an example, suppose there exists a minimalistic class for triangles called \lstinline|Triangle|,
which provides access to its vertices. For objects of this triangle class, color information modelled by a class \lstinline|RGBColor| should be stored within some application.
The following lines accomplish that for a triangle object \lstinline|tri| using {\ViennaData}:
\begin{lstlisting}
 //store RGB color information:
 viennadata::access<string, RGBColor>("color")(tri) = RGBColor(255,42,0);

 //get color information at some other point in the application:
 RGBColor col = viennadata::access<string, RGBColor>("color")(tri);
\end{lstlisting}
Data is associated with objects by keys of user-defined type in {\ViennaData}. In the example above,
the color information is stored using a key \lstinline|"color"| of type \lstinline|std::string|.
However, the user is free to use any key of any type\footnote{Some technical restrictions apply, see Sec.~\ref{sec:access}.}.
Thus, any associated data object \lstinline|data| of a generic type \lstinline|DataType| can be stored and accessed for an object of any type using a key \lstinline|key| of type \lstinline|KeyType|
by (cf.~Chapter \ref{chap:basic-usage})
\begin{lstlisting}
 //store 'data' for obj
 viennadata::access<KeyType, DataType>(key)(obj) = data;

 //retrieve the stored data:
 DataType data2 = viennadata::access<KeyType, DataType>(key)(obj);
\end{lstlisting}
Therefore, {\ViennaData} allows to design and work with classes which focus on their intrinsic properties only, and
ensures uniformity of associated data access over an arbitrarily large set of objects of different type.
This is in particular important, if several different libraries with different coding styles are used within a project.

With little additional configuration (cf.~Chapter \ref{chap:advanced-usage}) the code for color information access using {\ViennaData} can be reduced to
\begin{lstlisting}
 //store RGB color information
 viennadata::access<color_info>()(tri) = RGBColor(255, 42, 0);

 //get color information at some other point in the application:
 RGBColor col = viennadata::access<color_info>()(tri);
\end{lstlisting}
This enables uniform, type-safe data access throughout the whole application.
As shown in Chapter \ref{chap:benchmarks}, there is virtually no abstraction penalty for this gain in flexibility and uniformity. Explicit use cases are given in Chapter \ref{chap:use-cases}.

Internally, {\ViennaData} stores data in tree-like structures, for which the object address is used as key for storing the data.
However, if objects provide a means to identify themselves by numbers in the range $0, \ldots, N$ for some $N$, the internal datastructure of {\ViennaData} can be
conveniently customized to a more appropriate storage scheme. Similarly, if it is a-priori known that for a particular key type only one key object per key type is used, the
internal datastructure of {\ViennaData} can be adapted approporiately. Chapter \ref{chap:internals} explains the internals.

Finally, a few words are to be spent on the pitfalls and design decisions. Since data is associated with objects, object lifetimes have to be kept in mind.
If an object, for which associated data is stored, reaches the end of its lifetime, the user has to care for the associated data separately.
This can be very simple and fully transparent for the user in some cases, and rather tough in other cases. Moreover, {\ViennaData} may lead to unexpected results when used with polymorphism. More details are given in Chapter \ref{chap:pitfalls}, and some key design decisions are discussed in Chapter \ref{chap:design}.
