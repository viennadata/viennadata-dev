\chapter{Library Internals} \label{chap:internals}
In order to configure {\ViennaData} in the best way for the respective project and to avoid common pitfalls,
it is of advantage to understand some of the library-internals.
The focus is on explaining the \emph{big picture} rather than in going through all the details.
These details can either be found in the Doxygen generated documentation, or directly in the source code.

\section{Main Classes}
All data resides in the instances of the template class
\begin{lstlisting}
 template <typename KeyType,
           typename ValueType,
           typename ObjectType>
 class data_container;
\end{lstlisting}
located in \lstinline|data_container.hpp|.
As can be seen from the list of template parameters, each triple of key type, data type and object type lets
the compiler create another instance of \lstinline|data_container|. Thus, compilation times will increase with the number of different types involved.

\lstinline|data_container| uses a singleton pattern and is directly interfaced by the interface functions
\begin{lstlisting}
 viennadata::access<KeyType, ValueType>()()
 viennadata::copy<KeyType,   ValueType>()()
 viennadata::move<KeyType,   ValueType>()()
 viennadata::erase<KeyType,  ValueType>()()
 viennadata::find<KeyType,   ValueType>()()
\end{lstlisting}
The two parenthesis arguments are due to the following:
\begin{enumerate}
 \item The inner parenthesis belong to a function returning a proxy class
 \item The outer parenthesis denote the parenthesis operator \lstinline|operator()| of the proxy class
\end{enumerate}
Within the parenthesis operator, the appropriate functions of \lstinline|data_container| are called.

The storage type in \lstinline|data_container| is obtained from a traits class \lstinline|container_traits|, which provides the correct storage scheme.
By default, the storage is equivalent to
\begin{lstlisting}
 std::map<ObjectType *, std::map<KeyType, ValueType> >
\end{lstlisting}
Thus, the first call call
\begin{lstlisting}
 viennadata::access<KeyType, ValueType>(key)(obj) = data;
\end{lstlisting}
for \lstinline|obj| of type \lstinline|Object| is equivalent to
\begin{lstlisting}
 std::map<ObjectType *, std::map<KeyType, ValueType> >  storage;
 storage[&obj][key] = data;
\end{lstlisting}
and subsequent calls reuse the \lstinline|storage| object. If a dense storage scheme is used (cf.~Sec.~\ref{sec:dense-data} and \lstinline|storage_traits.hpp|), the storage type is equivalent to
\begin{lstlisting}
 std::vector< std::map<KeyType, ValueType> >
\end{lstlisting}
and the first call to \lstinline|viennadata::access| can be seen as
\begin{lstlisting}
 std::vector< std::map<KeyType, ValueType> >  storage;
 storage.resize(some_size);
 storage[obj.id()][key] = data;
\end{lstlisting}
The configuration of the ID retrieval (here: \lstinline|obj.id()|) is discussed in Sec.~\ref{sec:id-retrieval} and implemented in \lstinline|object_identifier.hpp|, while the need to call \lstinline|viennadata::reserve| in order to provide a value for \lstinline|some_size| is discussed in Sec.~\ref{sec:dense-data}. Finally, a compile time key dispatch (cf.~Sec.~\ref{sec:compiletime-keys}) eliminates the inner map of the storage scheme, hence the data is organized as
\begin{lstlisting}
 std::vector< ValueType >            dense_storage;
\end{lstlisting}
when using dense storage, and as
\begin{lstlisting}
 std::map<ObjectType *, ValueType >  sparse_storage;
\end{lstlisting}
for sparse storage. The traits classes in \lstinline|data_container_taits.hpp| ensure that the correct access mechanism is used for the various operations.

\section{viennadata::all} \label{sec:viennadata-all}
In order to make the following code lines
\begin{lstlisting}
 viennadata::access<std::string, long>("foo")(obj1) = 42;
 viennadata::access<long,        double>(360)(obj1) = 3.1415;
 viennadata::copy<viennadata::all, viennadata::all>()(obj1, obj2);
\end{lstlisting}
work as expected, a number of coding tricks have to be applied. As outlined in the previous section, the first two lines create two storage objects
\begin{lstlisting}
 std::map<ObjectType *, std::map<std::string, long> >
 std::map<ObjectType *, std::map<long, double> >
\end{lstlisting}
for which the data needs to be transferred. However, there is no means to iterate over compiler generated types, therefore the generated types have to be tracked at runtime.

A technique called \emph{type erasure} [CITE] is used to store pairs of different (!) \lstinline|KeyType|s and \lstinline|ValueType|s in an array, see \lstinline|key_value_pair.hpp| for the type erasure and \lstinline|key_value_registration.hpp| for the array. When using the \lstinline|viennadata::all| type for copy, move or erase operations, the type-erasured \lstinline|KeyType|s and \lstinline|ValueType|s are iterated and compared. If the types match, a virtual function call unwraps the types and calls the appropriate \lstinline|data_container|-member. This virtual function call is the reason why the source and the destination object have to be of the same type.

