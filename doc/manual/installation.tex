\chapter{Installation}

This chapter shows how {\ViennaData} can be integrated into a project and how
the examples are built. The necessary steps are outlined for several different
platforms, but we could not check every possible combination of hardware,
operating system and compiler. If you experience any trouble, please write to
the maining list at \\
\begin{center}
\texttt{viennadata-support$@$lists.sourceforge.net} 
\end{center}


% -----------------------------------------------------------------------------
% -----------------------------------------------------------------------------
\section{Dependencies}
% -----------------------------------------------------------------------------
% -----------------------------------------------------------------------------
\label{dependencies}
{\ViennaData} uses the {\CMake} build system for multi-platform support.
Thus, before you proceed with the installation of {\ViennaData}, make sure you
have a recent version of {\CMake} installed.

\begin{itemize}
 \item A recent C++ compiler (e.g.~{\GCC} version 4.2.x or above and Visual C++
2008 are known to work)
 \item {\CMake}~\cite{cmake} as build system (optional, but highly recommended
for building the examples)
\end{itemize}


\section{Generic Installation of ViennaData} \label{sec:viennacl-installation}
Since {\ViennaData} is a header-only library, it is sufficient to copy the
folder
\lstinline|viennadata/| either into your project folder or to your global system
include path. On Unix based systems, this is often \lstinline|/usr/include/| or
\lstinline|/usr/local/include/|.

On Windows, the situation strongly depends on your development environment. We
advise users
to consult the documentation of their compiler on how to set the include
path correctly. With Visual Studio this is usually something like
\texttt{C:$\setminus$Program Files$\setminus$Microsoft Visual Studio
9.0$\setminus$VC$\setminus$include}
and can be set in \texttt{Tools -> Options -> Projects and Solutions ->
VC++-\-Directories}. 


% -----------------------------------------------------------------------------
% -----------------------------------------------------------------------------
\section{Building the Examples and Tutorials}
% -----------------------------------------------------------------------------
% -----------------------------------------------------------------------------
For building the examples, we suppose that {\CMake} is properly set up
on your system. The other dependencies are listed in
Tab.~\ref{tab:tutorial-dependencies}.

\begin{table}[tb]
\begin{center}
\begin{tabular}{l|l}
Tutorial No. & Dependencies\\
\hline
\texttt{tutorial/tut1.cpp}      & {\OpenCL} \\
\texttt{tutorial/tut2.cpp}      & {\OpenCL}, {\ublas} \\
\texttt{tutorial/tut3.cpp}      & {\OpenCL}, {\ublas} \\
\texttt{tutorial/tut4.cpp}      & {\ublas} \\
\texttt{tutorial/tut5.cpp}      & {\OpenCL} \\
\texttt{benchmarks/vector.cpp}  & {\OpenCL} \\
\texttt{benchmarks/sparse.cpp}  & {\OpenCL}, {\ublas} \\
\texttt{benchmarks/solver.cpp}  & {\OpenCL}, {\ublas} \\
\end{tabular}
\caption{Dependencies for the examples in the \texttt{examples/} folder}
\label{tab:tutorial-dependencies}
\end{center}
\end{table}

\subsection{Linux}
To build the examples, open a terminal and change to:

\begin{exaipd}
\begin{Verbatim}
$> cd /your-ViennaData-path/build/
\end{Verbatim}
\end{exaipd}

Execute

\begin{exaipd}
\begin{Verbatim}
$> cmake ..
\end{Verbatim}
\end{exaipd}

to obtain a Makefile.
Executing

\begin{exaipd}
\begin{Verbatim}
$> make 
\end{Verbatim}
\end{exaipd}

builds the examples. If some of the dependencies in
Tab.~\ref{tab:tutorial-dependencies} are not fulfilled, you can build each
example separately:
\begin{exaipd}
\begin{Verbatim}
$> make tut1              #builds tutorial 1
$> make vectorbench       #builds vector benchmarks
\end{Verbatim}
\end{exaipd}


\TIP{Speed up the building process by using jobs, e.g. \keyword{make -j4}.}

\subsection{Mac OS X}
\label{apple}
The tools mentioned in Section \ref{dependencies} are available on 
macintosh platforms too. 
For the {\GCC} compiler the Xcode~\cite{xcode} package has to be installed.
To install {\CMake} and {\Boost} external portation tools have to be used, 
for example, Fink~\cite{fink}, DarwinPorts~\cite{darwinports} 
or MacPorts~\cite{macports}. Such portation tools provide the 
aforementioned packages, {\CMake} and {\Boost}, for macintosh platforms. 

\TIP{If the {\CMake} build system has problems detecting your {\Boost}
libraries, 
determine the location of your {\Boost} folder. 
Open the \texttt{CMakeLists.txt} file in the root directory of {\ViennaData}
and 
add your {\Boost} path after the following entry: 
\texttt{IF(\${CMAKE\_SYSTEM\_NAME} MATCHES "Darwin")} }

The build process of {\ViennaData} is similar to Linux.

\subsection{Windows}
In the following the procedure is outlined for \texttt{Visual Studio}: Assuming
that an {\OpenCL} SDK and {\CMake} is already installed, Visual Studio solution
and project files can be created using {\CMake}:
\begin{itemize}
\item Open the {\CMake} GUI.
\item Set the {\ViennaData} base directory as source directory.
\item Set the \texttt{build/} directory as build directory.
\item Click on 'Configure' and select the appropriate generator
(e.g.~\texttt{Visual Studio 9 2008})
\item Click on 'Generate' (you may need to click on 'Configure' one more time
before you can click on 'Generate')
\item The project files can now be found in the {\ViennaData} build directory,
where they can be opened and compiled with Visual Studio (provided that the
include and library paths are set correctly, see
Sec.~\ref{sec:viennacl-installation}).
\end{itemize}

\TIP{The examples and tutorials should be executed from within the
\texttt{build/} directory of {\ViennaData}, otherwise the sample data files
cannot be found.}

























