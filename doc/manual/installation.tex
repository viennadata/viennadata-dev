\chapter{Installation}

This chapter shows how {\ViennaData} can be integrated into a project and how
the examples are built. The necessary steps are outlined for several different
platforms, but we could not check every possible combination of hardware,
operating system, and compiler. If you experience any trouble, please write to
the mailing list at \\
\begin{center}
\texttt{viennadata-support$@$lists.sourceforge.net} 
\end{center}


% -----------------------------------------------------------------------------
% -----------------------------------------------------------------------------
\section{Dependencies}
% -----------------------------------------------------------------------------
% -----------------------------------------------------------------------------
\label{dependencies}

\begin{itemize}
 \item A recent C++ compiler (e.g.~{\GCC} version 4.2.x or above and Visual C++
2008 are known to work)
 \item {\CMake}~\cite{cmake} as build system (optional, but recommended
for building the examples)
\end{itemize}


\section{Generic Installation of ViennaData} \label{sec:viennacl-installation}
Since {\ViennaData} is a header-only library, it is sufficient to copy the 
\lstinline|viennadata/| folder either into your project folder or to your global system
include path. On Unix based systems, this is often \lstinline|/usr/include/| or
\lstinline|/usr/local/include/|.

On Windows, the situation strongly depends on your development environment. We
advise to consult the documentation of the compiler on how to set the include
path correctly. With Visual Studio this is usually something like
\texttt{C:$\setminus$Program Files$\setminus$Microsoft Visual Studio
9.0$\setminus$VC$\setminus$include}
and can be set in \texttt{Tools -> Options -> Projects and Solutions ->
VC++-\-Directories}. 


% -----------------------------------------------------------------------------
% -----------------------------------------------------------------------------
\section{Building the Examples and Tutorials}
% -----------------------------------------------------------------------------
% -----------------------------------------------------------------------------
For building the examples, we suppose that {\CMake} is properly set up
on your system. The other dependencies are listed in
Tab.~\ref{tab:tutorial-dependencies}.

\begin{table}[tb]
\begin{center}
\begin{tabular}{l|l}
File & Purpose\\
\hline
\texttt{tutorial/access.cpp}    & Shows the use of \lstinline|access| \\
\texttt{tutorial/erase.cpp}     & Shows the use of \lstinline|erase| \\
\texttt{tutorial/find.cpp}      & Shows the use of \lstinline|find| \\
\texttt{tutorial/copy.cpp}      & Shows the use of \lstinline|copy| \\
\texttt{tutorial/move.cpp}      & Shows the use of \lstinline|move| \\
\texttt{benchmarks/sparse.cpp}  & Benchmarks for different sparse storage schemes \\
\texttt{benchmarks/dense.cpp}   & Benchmarks for dense storage schemes \\
\end{tabular}
\caption{Dependencies for the examples in the \texttt{examples/} folder}
\label{tab:tutorial-dependencies}
\end{center}
\end{table}

\subsection{Linux}
To build the examples, open a terminal and change to:

\begin{lstlisting}
 $> cd /your-ViennaData-path/build/
\end{lstlisting}
Execute
\begin{lstlisting}
 $> cmake ..
\end{lstlisting}
to obtain a Makefile and type
\begin{lstlisting}
 $> make 
\end{lstlisting}
to build the examples. If desired, one can build each example separately instead:
\begin{lstlisting}
 $> make access            #builds the access tutorial
 $> make dense             #builds dense storage benchmark
\end{lstlisting}

\TIP{Speed up the building process by using jobs, e.g. \keyword{make -j4}.}

\subsection{Mac OS X}
\label{apple}
The tools mentioned in Section \ref{dependencies} are available on 
Macintosh platforms too. 
For the {\GCC} compiler the Xcode~\cite{xcode} package has to be installed.
To install {\CMake}, external portation tools such as
Fink~\cite{fink}, DarwinPorts~\cite{darwinports}, 
or MacPorts~\cite{macports} have to be used. 

The build process of {\ViennaData} is similar to Linux.

\subsection{Windows}
In the following the procedure is outlined for \texttt{Visual Studio}: Assuming
that an {\OpenCL} SDK and {\CMake} is already installed, Visual Studio solution
and project files can be created using {\CMake}:
\begin{itemize}
\item Open the {\CMake} GUI.
\item Set the {\ViennaData} base directory as source directory.
\item Set the \texttt{build/} directory as build directory.
\item Click on 'Configure' and select the appropriate generator
(e.g.~\texttt{Visual Studio 9 2008})
\item Click on 'Generate' (you may need to click on 'Configure' one more time
before you can click on 'Generate')
\item The project files can now be found in the {\ViennaData} build directory,
where they can be opened and compiled with Visual Studio (provided that the
include and library paths are set correctly, see
Sec.~\ref{sec:viennacl-installation}).
\end{itemize}

























