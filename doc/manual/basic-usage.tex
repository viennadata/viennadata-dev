\chapter{Basic Usage} \label{chap:basic-usage}

In this Section the fundamental operations in {\ViennaData} are covered in full detail. 
All functionality is available out of the box and operations are checked. If {\ViennaData} access turns out to be 
performance critical, several optimizations can be triggered, which is discussed in Chapter \ref{chap:advanced-usage}.

\section{Access}
The main function in {\ViennaData} is the \lstinline|access()| function. From a user perspective,
the call
\begin{lstlisting}
 viennadata::access<KeyType, ValueType>(key)(obj) 
\end{lstlisting}
returns a reference to the data of type \lstinline|ValueType| stored for an arbitrary object \lstinline|obj| and identified by the \lstinline|key| object of
type \lstinline|KeyType|. At first sight, the offered functionality can be seen as a \lstinline|std::map<KeyType, ValueType>| for each object \lstinline|obj|.
However, the \lstinline|access| function offers a lot more flexibility, which will be discussed in the following.

As a first example, to store RGB color information on objects \lstinline|obj1|, \lstinline|obj2| and \lstinline|obj3| using a \lstinline|std::string|, say \lstinline|"color"|, as key,
the code
\begin{lstlisting}
 viennadata::access<string, RGBColor>("color")(obj1) = RGBColor(255,42,0);
 viennadata::access<string, RGBColor>("color")(obj2) = RGBColor(0,42,255);
 viennadata::access<string, RGBColor>("color")(obj3) = RGBColor(10,20,33);
\end{lstlisting}
is sufficient. Note that \lstinline|obj1|, \lstinline|obj2|, and \lstinline|obj3| can be of different type.
The data can be recovered directly via
\begin{lstlisting}
 RGBColor col1 = viennadata::access<string, RGBColor>("color")(obj1);
 RGBColor col2 = viennadata::access<string, RGBColor>("color")(obj2);
 RGBColor col3 = viennadata::access<string, RGBColor>("color")(obj3);
\end{lstlisting}
In addition, several different colors can be associated with a single object \lstinline|obj1|
as
\begin{lstlisting}
 viennadata::access<string, RGBColor>("front")(obj1) = RGBColor(255,42,0);
 viennadata::access<string, RGBColor>("back")(obj1) = RGBColor(12,21,255);
 viennadata::access<string, RGBColor>("side")(obj1) = RGBColor(30,20,10);
\end{lstlisting}
and similarly for \lstinline|obj2| and \lstinline|obj3| if desired.
 Since any type that fulfills the \lstinline|LessThan-Comparable| concept
\footnote{\lstinline|operator<| can be used for comparisons} [CITE STL] can be used as key type, one can also use keys of different types, e.g.
\begin{lstlisting}
 viennadata::access<string, RGBColor>("front")(obj1) = RGBColor(255,42,0);
 viennadata::access<long,   RGBColor>( 12345 )(obj1) = RGBColor(12,21,255);
 viennadata::access<double, RGBColor>( 42.42 )(obj1) = RGBColor(30,20,10);
\end{lstlisting}
If color is to be saved in different color spaces, this could be accomplished by adjusting the data type accordingly:
\begin{lstlisting}
 viennadata::access<string, RGBColor>("front")(obj1) = RGBColor(255,42,0);
 viennadata::access<long,   HSVColor>( 12345 )(obj1) = HSVColor(12,21,255);
 viennadata::access<double, CMYKColor>(42.42 )(obj1) = CMYKColor(0,1,2,3);
\end{lstlisting}
At this point, the second template argument appears to be somewhat redundant.
However, please read about the design decisions in Chapter \ref{chap:design} in order to get more details on why the second template paramter is still required.


In the last examples, only one key object per key type has been used. As will be shown in 
Chapter \ref{chap:advanced-usage}, one can completely turn off run-time comparisions of keys,
so that one can write
\begin{lstlisting}
 viennadata::access<front_color, RGBColor>()(obj1) = RGBColor(255,42,0);
 viennadata::access<back_color,  HSVColor>()(obj1) = HSVColor(12,21,255);
 viennadata::access<side_color, CMYKColor>()(obj1) = CMYKColor(0,1,2,3);
\end{lstlisting}
where \lstinline|front_color|, \lstinline|back_color| and \lstinline|side_color| are empty tag classes [CITE] that are solely used as a key type.
By using the types of the key only, the run time key comparisons are modified to compile time dispatches. This results in considerably faster
execution, provided that it is already known at compile time which key has to be used.

\NOTE{Compile time dispatches are disabled by default. On details how to enable them, refer to Sec.~\ref{sec:compiletime-keys}}

With compile time dispatches enabled, the stored data can be retained in the same way:
\begin{lstlisting}
 RGBColor  col1 = viennadata::access<front_color, RGBColor>()(obj1);
 HSVColor  col2 = viennadata::access<back_color,  HSVColor>()(obj1);
 CMYKColor col3 = viennadata::access<side_color, CMYKColor>()(obj1);
\end{lstlisting}
For the case that it is a-priori known that for each \lstinline|KeyType| only one \lstinline|ValueType| will be used, {\ViennaData}
can be configured to explicitly bind the \lstinline|KeyType| to a particular \lstinline|ValueType|. In the previous code snippet,
the \lstinline|front_color| can be bound to the \lstinline|RGBColor| type, and similarly the \lstinline|back_color| to \lstinline|HSVColor|
and \lstinline|side_color| to \lstinline|CMYKColor|. In this case, the previous code snippet reduces to
\begin{lstlisting}
 RGBColor  col1 = viennadata::access<front_color>()(obj1);
 HSVColor  col2 = viennadata::access<back_color>()(obj1);
 CMYKColor col3 = viennadata::access<side_color>()(obj1);
\end{lstlisting}
Similarly, no \lstinline|ValueType| is used for storing data in this case.

\NOTE{See Sec.~\ref{sec:default-valuetype} for details on how to specify a default \lstinline|ValueType| for a particular \lstinline|KeyType|.}


\section{Copy} \label{sec:copy}
In order to copy data of type \lstinline|ValueType| stored using a \lstinline|key| of type \lstinline|KeyType| from an object \lstinline|obj_src| to an object \lstinline|obj_dest|,
the \lstinline|copy()| function can be used as follows;
\begin{lstlisting}
 viennadata::copy<KeyType, ValueType>(key)(obj_src, obj_dest);
\end{lstlisting}
In this case, only the particular data for the particular key is copied. If the \lstinline|key| argument is omitted, i.e.
\begin{lstlisting}
 viennadata::copy<KeyType, ValueType>()(obj_src, obj_dest);
\end{lstlisting}
all data stored for \lstinline|obj_src| using any key of type \lstinline|KeyType| is copied to \lstinline|obj_dest|.
As an example, consider
\begin{lstlisting}
 viennadata::access<std::string, RGBColor>("front")(obj_src) = RGBColor(255,0,0);
 viennadata::access<std::string, RGBColor>("back")(obj_src) = RGBColor(0,255,0);
 viennadata::access<std::string, RGBColor>("side")(obj_src) = RGBColor(0,0,255);

 //copy all data from obj_src to obj_dest:
 viennadata::access<std::string, RGBColor>()(obj_src, obj_dest);

 //access the front color (result will be RGBColor(255, 0, 0))
 RGBColor red = viennadata::access<std::string, RGBColor>("front")(obj_dest);
\end{lstlisting}
In this example, the colors stored for \lstinline|obj_src| are copied to \lstinline|obj_dest| and can then be accessed using the keys \lstinline|"front"|, \lstinline|"back"| and \lstinline|"side"|.

\TIP{\lstinline|obj_src| and \lstinline|obj_dest| do not need to be of the same type for basic \lstinline|copy()| functionality!}

\NOTE{For technical reasons, \lstinline|obj_src| and \lstinline|obj_dest| have to be of the same type if \lstinline|all| is used.}

So far, every pair of \lstinline|KeyType| and \lstinline|ValueType| has to be copied by a separate function call.
For larger projects where multiple different keys and data types are used, this is too tedious. To copy all
data of possibly different type for a particular key, one can use the dedicated \lstinline|all| specifier if :
\begin{lstlisting}
 viennadata::copy<KeyType, viennadata:all>()(obj_src, obj_dest);
\end{lstlisting}
Similarly, to copy all data of type \lstinline|ValueType| stored for \lstinline|obj_src| to \lstinline|obj_dest|, the line
\begin{lstlisting}
 viennadata::copy<viennadata:all, ValueType>()(obj_src, obj_dest);
\end{lstlisting}
can be used. Finally, to copy all data of arbitrary type stored using any key of any type for \lstinline|obj_src| to \lstinline|obj_dest|,
the line
\begin{lstlisting}
 viennadata::copy<viennadata:all, viennadata:all>()(obj_src, obj_dest);
\end{lstlisting}
is sufficient.

\NOTE{In order to have \lstinline|viennadata:all| working correctly, every pair of \lstinline|KeyType| and \lstinline|ValueyType| in use must be registered.
By default, this is done automatically in the background. 
If the library user decides to disable automatic registration, each such pair must be registered manually (cf.~Section \ref{sec:register-data}).}

\section{Move}
Data of type \lstinline|ValueType| stored using a \lstinline|key| of type \lstinline|KeyType| for an object \lstinline|obj_src| can be transferred to another object \lstinline|obj_dest| by using
the \lstinline|move()| function as follows:
\begin{lstlisting}
 viennadata::move<KeyType, ValueType>(key)(obj_src, obj_dest);
\end{lstlisting}
The difference to the \lstinline|copy()| function in the previous section is that the data with \lstinline|move()| the data is then only available for \lstinline|obj_dest|.

Similar to \lstinline|copy()|, the following variants are available:
\begin{lstlisting}
 viennadata::move<KeyType, ValueType>()(obj_src, obj_dest);
 viennadata::move<KeyType, viennadata:all>()(obj_src, obj_dest);
 viennadata::move<viennadata:all, ValueType>()(obj_src, obj_dest);
 viennadata::move<viennadata:all, viennadata:all>()(obj_src, obj_dest);
\end{lstlisting}
Since their use is similar to that of the \lstinline|copy()| variants, the reader is referred to the previous section.

\NOTE{For technical reasons, \lstinline|obj_src| and \lstinline|obj_dest| have to be of the same type if \lstinline|all| is used.}

\section{Erase}
In order to delete data of type \lstinline|ValueType| stored using a \lstinline|key| of type \lstinline|KeyType| for an object \lstinline|obj|,
the \lstinline|erase()| function can be used as follows:
\begin{lstlisting}
 viennadata::erase<KeyType, ValueType>(key)(obj);
\end{lstlisting}
Similar to \lstinline|copy()| and \lstinline|move()|, the following variants are available:
\begin{lstlisting}
 viennadata::erase<KeyType, ValueType>()(obj);
 viennadata::erase<KeyType, viennadata:all>()(obj);
 viennadata::erase<viennadata:all, ValueType>()(obj);
 viennadata::erase<viennadata:all, viennadata:all>()(obj);
\end{lstlisting}
They allow to delete larger amounts of data associated with \lstinline|obj| by a single call.
For more explainations, the reader is referred to Section \ref{sec:copy}.

\section{Query Data Availability}
\lstinline|viennadata::find()| allows to check whether data is stored for a particular object. If there is data stored, a pointer to the data is returned. Otherwise, a \lstinline|NULL| pointer is returned.
the following code snippets shows how to check for data of type \lstinline|ValueType| associated with \lstinline|obj| using a \lstinline|key| of type \lstinline|KeyType|:
\begin{lstlisting}
 if (viennadtaa::find<KeyType, ValueType>(key)(obj))
   /* data available */
 else
   /* data not yet available */
\end{lstlisting}
\lstinline|find| is very attractive if there is a large number of objects with only a few carrying data.
Using \lstinline|viennadata::access| to query for data existence by iterating through all objects would create data of type \lstinline|ValueType| for every object. This overhead is avoided with \lstinline|viennadata::find|.

